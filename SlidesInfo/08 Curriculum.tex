\documentclass[handout]{beamer}
\usepackage[T1]{fontenc}
\usepackage[utf8]{inputenc}
\usepackage{lmodern}
\usepackage[italian]{babel}

\title{Creazione di un curriculum}
\author{Mattia Cozzi}
\date{a.f.~2024/2025}


%\documentclass[handout]{beamer}     %usare questa classe per generare l'handout

%\usepackage{pdfpages}   %per mostrare più quadri nella stessa pagina
%\pgfpagesuselayout{4 on 1}[a4paper,border shrink=5mm,landscape]


\usetheme{Singapore}
%\useoutertheme[left]{sidebar} %elementi intorno alle diapositive
\setbeamercovered{dynamic} %modifica l'aspetto del testo grigetto delle diapositive future. Argomenti: invisible/transparent/dynamic


%COLORE PRINCIPALE
\definecolor{verde}{RGB}{2, 194, 117} % UBC Blue (primary)
\setbeamercolor{structure}{fg=verde} % itemize, enumerate, etc
\setbeamercolor{alerted text}{fg=verde}


\usecolortheme{orchid}

\usepackage{tikz}

\begin{document}

\begin{frame}
  \titlepage
\end{frame}


\begin{frame}
\frametitle{Contenuti}
\tableofcontents
\end{frame}



\section{Introduzione}


\begin{frame}
\frametitle{Ricerca di personale}
Quando un'azienda è alla ricerca di personale \alert{può muoversi in maniera diversa} a seconda della figura ricercata, della propria cultura aziendale ed della propria policy di selezione e gestione del personale.\pause

~

Le aziende medio-grandi hanno un reparto che si occupa di selezione e gestione del personale (\alert{HR}, \emph{human resources}).\pause

~

Nelle aziende più piccole (ad esempio a conduzione familiare) può essere direttamente il/la titolare o un suo incaricato/a a gestire la fase di selezione.
\end{frame}

\begin{frame}
\frametitle{Modalità di ricerca di personale (1)}
Esistono diversi modi di trovare nuovi lavoratori e lavoratrici:
\begin{itemize}
  \item \alert{informalmente}, attivando la propria rete di relazioni, chiedendo di segnare nominativi di persone valide;{\pause} in questi casi può prevalere il rapporto di fiducia con chi consiglia il personale, a scapito della valutazione delle competenze.
\end{itemize}
\visible<2->{\begin{figure}
  \includegraphics[width=.6\columnwidth]{img/conversazione.jpg}
\end{figure}}
\end{frame}

\begin{frame}
\frametitle{Modalità di ricerca di personale (2)}
\begin{itemize}\setcounter{enumi}{1}
  \item con \alert{banche dati interne e sito aziendale}, cioè cercando tra i curricula che ricevono (\alert{candidatura spontanea}) o attivando la sezione ``Carriere/Lavora con noi'' del loro sito web.
\end{itemize}
\begin{figure}
  \includegraphics[width=.8\columnwidth]{img/lavoraconnoi.png}
\end{figure}
\end{frame}


\begin{frame}
\frametitle{Modalità di ricerca di personale (3)}
\begin{itemize}\setcounter{enumi}{1}
  \item con \alert{annunci su giornali e portali online}, come \href{https://www.linkedin.com/}{LinkedIn}, \href{https://it.indeed.com/}{Indeed} o \href{https://www.infojobs.it/}{Infojobs}; in questi casi bisogna essere molto attenti ad evitare truffe e raggiri, diffidare di chi offre guadagni rapidi e da annunci senza informazioni.
\end{itemize}
\begin{figure}
  \includegraphics[width=.8\columnwidth]{img/indeed.png}
\end{figure}
\end{frame}



\begin{frame}
\frametitle{Modalità di ricerca di personale (4)}
\begin{itemize}\setcounter{enumi}{1}
  \item con \alert{intermediari}, come centri per l'impiego, centri di formazione professionale o agenzie di ricerca e selezione personale;{\pause} queste ultime sono dette anche \alert{agenzie interinali}, come Adecco, Randstad, Etjca, ecc.
\end{itemize}
\visible<2->{\begin{figure}
  \includegraphics[width=.8\columnwidth]{img/adecco.png}
\end{figure}}
\end{frame}



\begin{frame}
\frametitle{Cos'è il curriculum}
Il termine \alert{curriculum vitae} viene dal latino e significa ``percorso di vita'': serve a presentare la propria situazione personale, scolastica e lavorativa.\pause

~

È una sorta di ``biglietto da visita professionale'', con cui proporsi per un posto e partecipare alla selezione di personale.\pause

~

È la prima cosa che viene richiesta durante la ricerca di un lavoro, ed è ciò che usiamo per \alert{arrivare ad un colloquio di lavoro}.
\end{frame}


\begin{frame}
\frametitle{Curriculum efficace e non efficace}
È importante preparare CV \alert{mirati e personalizzati} in base all'azienda che stiamo contattando e per il profilo per il quale ci stiamo candidando.\pause

~

Con un buon CV dobbiamo:
\begin{itemize}
  \item comunicare informazioni \alert{interessanti};\pause
  \item comunicare informazioni \alert{pertinenti};\pause
  \item \alert{incuriosire} il lettore.
\end{itemize}
\end{frame}


\begin{frame}
\frametitle{Caratteristiche fondamentali di un buon CV}
Un buon curriculum deve essere:
\begin{itemize}
  \item \alert{completo}, con tutte le informazioni utili su di noi, sia lavorativamente sia sulla nostra personalità;\pause
  \item \alert{sintetico}, evitando frasi ridondanti, lungo al massimo due pagine;\pause
  \item \alert{chiaro e comunicativo}, scritto al computer e facendo risaltare le informazioni più interessanti;\pause
  \item \alert{mirato}, scritto in modo da avvicinarsi il più possibile, senza mentire, alla posizione per cui ci candidiamo.
\end{itemize}
\end{frame}

\section{Contenuto}


\begin{frame}
\frametitle{Dati da indicare necessariamente}
\begin{itemize}
  \item Nome e cognome;\pause
  \item residenza;\pause
  \item numero di telefono e indirizzo email;\pause
  \item possesso patente/i e mezzi di trasporto;\pause
  \item percorso di istruzione e formazione (quanto basta);\pause
  \item esperienze professionali precedenti, con riferimenti specifici a luoghi e date;\pause
  \item stage e tirocini;\pause
  \item conoscenze tecniche (\emph{hard skills});\pause
  \item conoscenze linguistiche (con livello di padronanza);\pause
  \item conoscenze informatiche (con livello di padronanza);\pause
  \item competenze trasversali (\emph{soft skills}: capacità di organizzare il lavoro, rapporto coi colleghi, ecc.).
\end{itemize}
\end{frame}

\begin{frame}
\frametitle{Devo scrivere tutto?}
Dati da riportare se richiesto:
\begin{itemize}
  \item data di nascita;\pause
  \item fotografia (primo piano, sfondo neutro).\pause
\end{itemize}

~

Dati da non riportare:
\begin{itemize}
  \item segno zodiacale;\pause
  \item informazioni inutili su di noi;\pause
  \item dati fisici, a meno che non siano richiesti per lavoro (es.~modella).\pause
\end{itemize}

~

Non ci devono essere errori grammaticali o di battitura!\pause

~

Attenzione a riportare i propri contatti social, può essere un'arma a doppio taglio!
\end{frame}


\begin{frame}
\frametitle{Esperienze non lavorative}
Inserire le esperienze non lavorative nel curriculum a volte può \alert{fare la differenza}.\pause

~

Aver partecipato ad incontri, corsi di formazione, attività di volontariato ed altre esperienze attinenti al ruolo che si andrà a ricoprire potrebbe essere decisivo per un'assunzione.\pause

~

Di fronte ad un altro profilo con un percorso professionale e formativo pari o simile al nostro, il selezionatore potrebbe decidere di darci un'occasione proprio in virtù della nostra \alert{esperienza al di fuori del lavoro}.
\end{frame}



\begin{frame}
\frametitle{Soft skills}
Alcuni esempi di \emph{soft skills}:
\begin{itemize}
  \item essere creativi/e;\pause
  \item essere in grado di prendere decisioni;\pause
  \item possedere l'intelligenza emotiva, cioè saper ascoltare e capire come stanno le persone intorno a noi;\pause
  \item saper fare Problem Solving, cioè ragionare per risolvere problemi;\pause
  \item capacità di gestire un gruppo;\pause
  \item saper gestire lo stress;\pause
  \item essere proattivo/a, cioè saper prendere iniziative senza subire le decisioni degli altri.
\end{itemize}
  

\end{frame}


\begin{frame}
\frametitle{Consenso al trattamento dei dati}
Uno degli errori più comuni frequenti nella scrittura di un CV è quello di \alert{dimenticarsi di inserire l'autorizzazione al trattamento dei dati personali}.\pause

~

Inserirla significa autorizzare chi riceve il CV ad analizzare ed usare i dati del candidato/a, che sono tutelati dalla legge sulla privacy.\pause

~

In sostanza, senza quella dicitura legalmente le aziende non potrebbero contattarti. È quindi importante ricordarsi di inserirla in basso nell’ultima pagina del CV, con data e firma.
\end{frame}


\begin{frame}
\frametitle{Dicitura standard per il trattamendo dei dati}
Esistono diverse formule per acconsentire al trattamento dei dati personali, come ad esempio:

~

\begin{quote}
  Autorizzo il trattamento dei dati personali contenuti nel mio curriculum vitae in base al \alert{Dlgs 196 del 30 giugno 2003} e all’\alert{art. 13 del GDPR}.
\end{quote}\pause

~

~

Ricorda infine sempre di \alert{datare e firmare il CV}!
\end{frame}



\section{Europass}

\begin{frame}
\frametitle{Europass}
\begin{figure}
  \includegraphics[width=.35\columnwidth]{img/europass.png}
\end{figure}
Il Curriculum Vitae europeo, o Europass, è un formato introdotto a partire dal 2002 dalla Commissione Europea. È un \alert{modello standard} e permette una lettura rapida e chiara del CV per poterlo confrontare velocemente con altri.\pause

~

Il CV europeo ha una struttura semplice e lineare ed è diviso in sezioni con un \alert{layout a colonne}.
\end{frame}


\begin{frame}
\frametitle{Esempio di CV in formato europeo}
\begin{figure}
  \includegraphics[width=.9\columnwidth]{img/cveuromc.png}
\end{figure}
\end{frame}

\begin{frame}
\frametitle{Vantaggi e svantaggi}
Il Curriculum Vitae europeo è il formato ideale per chi sta valutando l'idea di cercare lavoro all'estero e per chi non ha molte esperienze di lavoro da illustrare.\pause

~

Se invece hai molte esperienze lavorative alle spalle o un importante percorso di formazione, l'Europass potrebbe non essere il modello più indicato da utilizzare, poiché mette a disposizione poco spazio.\pause

~

Ha inoltre lo svantaggio di non essere particolarmente ``interessante'', quindi se stiamo cercando lavori più creativi può essere utile anche creare un CV di un altro tipo.
\end{frame}



\begin{frame}
\frametitle{CV europeo online}
Dal sito \href{https://europass.europa.eu/it}{https://europass.europa.eu/it} è possibile seguire una \alert{procedura guidata} per la creazione del proprio curriculum europeo.

~

\begin{figure}
  \includegraphics[width=.8\columnwidth]{img/cvonline.png}
\end{figure}
\end{frame}

\begin{frame}
\frametitle{Modelli online}
Per crearlo sul nostro computer esistono dei \alert{template preimpostati} e modificabili che si trovano facilmente su Internet.\pause

~

Con una rapida ricerca su Google di ``modello cv europass'' troviamo:
\begin{center}
  \href{https://www.governo.it/sites/governo.it/files/cvformatoeuropeoB.doc}{www.governo.it/sites/governo.it/files/cvformatoeuropeoB.doc}
\end{center}
che scarichiamo e possiamo modificare con Word.
\end{frame}

\begin{frame}
\frametitle{Template Europass modificabile}
\begin{figure}
  \includegraphics[width=\columnwidth]{img/cvtemplate.png}
\end{figure}

Il file scaricato è organizzato con delle \alert{caselle di testo}, che andremo a riempire coi nostri dati.
\end{frame}


\begin{frame}
\frametitle{Esercizi}
\begin{enumerate}
  \item Crea il tuo \alert{curriculum reale}, modificando il modello proposto in queste slides. Aggiungi anche la tua foto e i tuoi contatti. Salvalo in formato PDF, stampalo e firmalo.
  
  ~
  \item Crea il curriculum di un \alert{personaggio immaginario} che si sta candidando per la posizione di manager di un supermercato molto grande in provincia di Milano. Salvalo in formato PDF.
\end{enumerate}
\end{frame}


\section{Altri modelli}

\begin{frame}
\frametitle{Altri tipi di CV}
Il tuo Cv potrebbe assumere anche le sembianze diverse dal tradizionale formato del Cv o del
formato Europass: la creazione di un'infografica che contiene le informazioni del tuo profilo è un
modo originale e immediato di presentarti alle aziende, soprattutto nell'epoca dei Social Network.
Le Infografiche di Cv hanno il vantaggio di poter realizzare Cv creativi, chiari, diretti mettendo a
disposizione di chi legge (in un unico foglio) le informazioni che vuoi comunicare e mettere in
evidenza rispetto al posto per il quale ti stai candidando. I tuoi dati anagrafici, la tua formazione, le
tue esperienze e competenze saranno tradotti in un formato grafico: si tratta infatti di una
trasposizione grafica e visiva di informazioni. Potrai creare il tuo Cv infografico attraverso l'utilizzo di
alcune piattaforme online che mettono a disposizione gratuitamente dei layout grafici a questo
specifico scopo. Spesso questi siti traggono informazioni dai profili social degli utenti (ad esempio
Linkedin). Su Canva è possibile trovare modelli di CV infografica
  

\end{frame}


\begin{frame}
\frametitle{Canva}
\begin{figure}
  \includegraphics[width=.5\columnwidth]{img/cvinfografica.png}
\end{figure}
\end{frame}

\section{Presentazione}


% Ha lo scopo di indicare la ragione per cui si è inviato il curriculum vitae e deve evidenziare
% brevemente (non deve superare la mezza pagina) i punti forti di chi scrive, invogliando il
% selezionatore a leggere il curriculum vitae allegato. A seconda che si risponda ad un annuncio o ci si
% autocandidi, deve essere impostata in modo diverso:
% • Autocandidatura
% Quando ci si autocandida, la lettera d'accompagnamento deve far comprendere al
% selezionatore, con appropriato tatto, il tipo di lavoro che si gradirebbe ricoprire all’interno di
% quella azienda, i motivi dell’interessamento per quella tipologia di mansione e soprattutto i
% motivi per cui si potrebbe essere un utile collaboratore.
% • Risposta annuncio
% Nel caso di risposta ad un annuncio apparso su un quotidiano o su altri canali, il Curriculum
% Vitae deve indicare perché si è interessati a quella mansione e le caratteristiche personali
% utili a ricoprire il ruolo richiesto nell’annuncio.
% Può risultare utile inviare il proprio curriculum vitae anche se non vi è perfetta coincidenza
% tra quanto richiesto dall'azienda e le caratteristiche possedute: può darsi che l'azienda decida
% comunque di valutare la possibilità di impegnare il candidato. È importante citare la fonte
% tramite la quale si è venuti a conoscenza dell'opportunità.
% Grafica della lettera di presentazione
% La lettera di presentazione deve indicare: intestazione, luogo e data, destinatario, oggetto, testo e
% firma. Può essere consigliabile scriverla a mano (se si ha una buona calligrafia) in quanto ciò
% dimostrerebbe un interesse particolarmente sentito verso quell'azienda. In alternativa, si può
% utilizzare il computer ricordando di riportare sempre un destinatario, in questo modo, infatti, non si
% darà l’impressione che si stia mandando una quantità spropositata di curricula senza alcun interesse
% specifico. In ogni caso, dopo i saluti, la firma dovrà essere riportata in originale. Relativamente al
% destinatario, è opportuno che la lettera sia indirizzata ad una specifica persona: l'incaricato della
% selezione, il direttore del personale o l'ufficio selezione del personale. È importante ricordare di
% preparare un'intestazione, che può comparire anche a piè di pagina e che contenga tutti i recapiti
% utili per essere contattati.


\begin{frame}
\frametitle{Cos'è Google Calendar}
Google Calendar è l'\alert{agenda elettronica} della suite Google.\pause

~

\begin{columns}
\begin{column}{.7\textwidth}
  Permette di:
  \begin{itemize}
    \item gestire i propri appuntamenti;\pause
    \item gestire gli appuntamenti di un gruppo di persone;\pause
    \item programmare eventi ricorrenti;\pause
    \item programmare videochiamate;\pause
    \item creare eventi a cui invitare partecipanti esterni.
  \end{itemize}  
\end{column}
\begin{column}{.2\textwidth}
  \begin{figure}
    \includegraphics[width=\columnwidth]{img/calendarlogo.png}
  \end{figure}
\end{column}
\end{columns}
\end{frame}



\section{Portfolio}

\begin{frame}
\frametitle{Cos'è un portfolio?}

  

\end{frame}


\begin{frame}
\frametitle{Interfaccia principale}
Creare un sito web come portfolio!!
\end{frame}

\end{document}
