\documentclass[]{beamer}
\usepackage[T1]{fontenc}
\usepackage[utf8]{inputenc}
\usepackage{lmodern}
\usepackage[italian]{babel}
\usepackage{mathrsfs}

\title{I circuiti in corrente alternata}
\author{\texorpdfstring{Mattia Cozzi\newline\href{mailto:cozzimattia@gmail.com}{\texttt{cozzimattia@gmail.com}}}{Mattia Cozzi}}
\date{a.s.~2023/2024}

%\documentclass[handout]{beamer}     %usare questa classe per generare l'handout
%\usepackage{pgfpages}   %per mostrare più quadri nella stessa pagina
%\pgfpagesuselayout{4 on 1}[a4paper,border shrink=5mm,landscape]
\usetheme{Singapore}
%\useoutertheme[left]{sidebar} %elementi intorno alle diapositive
\setbeamercovered{dynamic} %modifica l'aspetto del testo grigetto delle diapositive future. Argomenti: invisible/transparent/dynamic
\usecolortheme{orchid}
%COLORE PRINCIPALE
% \definecolor{marroncino}{RGB}{156, 26, 0} % UBC Blue (primary)
% \setbeamercolor{structure}{fg=marroncino} % itemize, enumerate, etc


\theoremstyle{plain}
\newtheorem{teorema}{Teorema}

\usepackage{tikz}
\usepackage{circuitikz}

\usepackage{pgf,pgfplots,graphicx}
\usetikzlibrary{angles,quotes,arrows,shapes,decorations.markings}
\pgfplotsset{compat=1.15}
\usepgfplotslibrary{units,fillbetween} % to add units easily to axis

\newcommand{\fem}{f_{em}}
\newcommand{\femm}{f_{em0}}

\def\angolo[#1](#2)(#3:#4:#5)% Syntax: [draw options] (center) (initial angle:final angle:radius)
    { \draw[#1] ($(#2)+({#5*cos(#3)},{#5*sin(#3)})$) arc (#3:#4:#5); }


\begin{document}

\begin{frame}
  \titlepage
\end{frame}





\begin{frame}
\frametitle{Contenuti}
\tableofcontents
\end{frame}

\section{Premessa}

\begin{frame}
\frametitle{Alternatore}
\begin{figure}
\ctikzset{bipoles/length=1.2cm}
\begin{circuitikz}[scale=0.5]
\draw (4,0) to[vco, l=$\fem(t)$] (0,0);
\end{circuitikz}
\end{figure}
Un generatore di corrente alternata fornisce una $ \fem $ e una corrente variabili secondo le leggi:
\begin{center}
   \colorbox{blue!30}{$ \fem(t) = \femm \cdot \sin (\omega t) $}
~~~~~   \colorbox{blue!30}{$ i(t) = i_0 \cdot \sin (\omega t) $}
\end{center}
con:
\begin{itemize}
  \item $ \femm $ = forza elettromotrice massima;
  \item $ i_0 $ = intensità di corrente massima;
  \item $ \omega $ = pulsazione.
\end{itemize}
\end{frame}

\begin{frame}
\frametitle{Grafici di $ \fem(t) $ e $ i(t) $}


\begin{columns}
\begin{column}{0.5\textwidth}
\begin{figure}
\begin{tikzpicture}[xscale=.3,yscale=1]
\node [left] at (0,1) {{\scriptsize $ \femm $}};
\node [below] at (4*pi,0) {{\scriptsize $ t \, [s] $}};
\node [left] at (0,2) {{\scriptsize $ \fem(t) \, [V] $}};
\node [left] at (0,-1) {{\scriptsize $ - \femm $}};
\node [below] at (1*pi,-1) {{\scriptsize $ T=\dfrac{2\pi}{\omega} $}};
\draw [->] (-1,0) -- (4*pi,0);
\draw [|<->|, dashed, thick] (0,-1) -- (2*pi,-1);
\draw [-, dotted] (3.5*pi,1) -- (0*pi,1);
\draw [-, dotted] (0*pi,-1) -- (3.5*pi,-1);
\draw [-, dotted] (2*pi,-1) -- (2*pi,1);
\draw [->] (0,-2) -- (0,2);
\draw[smooth, cyan, thick, domain=0:3.5*pi] plot (\x, {sin(\x r)});
\end{tikzpicture}
\end{figure}
\end{column}
\begin{column}{0.5\textwidth}
\begin{figure}
\begin{tikzpicture}[xscale=.3,yscale=1]
\node [left] at (0,1) {{\scriptsize $ i_0 $}};
\node [below] at (4*pi,0) {{\scriptsize $ t \, [s] $}};
\node [left] at (0,2) {{\scriptsize $ i(t) \, [A] $}};
\node [left] at (0,-1) {{\scriptsize $ - i_0 $}};
\node [below] at (1*pi,-1) {{\scriptsize $ T=\dfrac{2\pi}{\omega} $}};
\draw [->] (-1,0) -- (4*pi,0);
\draw [|<->|, dashed, thick] (0,-1) -- (2*pi,-1);
\draw [-, dotted] (3.5*pi,1) -- (0*pi,1);
\draw [-, dotted] (0*pi,-1) -- (3.5*pi,-1);
\draw [-, dotted] (2*pi,-1) -- (2*pi,1);
\draw [->] (0,-2) -- (0,2);
\draw[smooth, red, thick, domain=0:3.5*pi] plot (\x, {sin(\x r)});
\end{tikzpicture}
\end{figure}
\end{column}
\end{columns}

~

~

I valori di $ i_0 $ e di $ \femm $ sono legati dalla prima legge di Ohm:
\[i_0 = \frac{\femm}{R}\]
\end{frame}



\begin{frame}
\frametitle{Induttanza}
\begin{figure}
\ctikzset{bipoles/length=1.2cm}
\begin{circuitikz}[scale=0.5]
\draw (0,0) to[L, a=$ f_{em ~auto} $] (4,0);
\end{circuitikz}
\end{figure}
Se in un circuito è presente un'induttanza avremo fenomeni di autoinduzione, ovvero una \alert{$ \fem $ autoindotta} data da:
\begin{center}
\colorbox{blue!30}{$ f_{em ~auto} = -L \cdot \dfrac{di}{dt} $}
\end{center}
con:\begin{itemize}
  \item $ L $ = induttanza [$ H $];
  \item $ \dfrac{di}{dt} $ = derivata temporale della corrente.
\end{itemize}
\end{frame}


\section{Ohmico}

\begin{frame}
\frametitle{Circuito ohmico (resistivo)}

\begin{figure}
\ctikzset{bipoles/length=.7cm}
\begin{circuitikz}[scale=0.5]
\draw (0,0) to[vco, a=$ \fem(t) $] (6,0);
\draw (0,4) to[short] (0,0);
\draw (0,4) to[R, l=$ R $] (6,4);
\draw (6,4) to[short] (6,0);
\end{circuitikz}
\end{figure}
Vale semplicemente la \alert{prima legge di Ohm}:
\begin{center}
\colorbox{blue!30}{$ i(t) = \dfrac{\fem(t)}{R} $}
\end{center}
\end{frame}


\begin{frame}
\frametitle{Grafico delle funzioni}
La fase di entrambe le onde è $ \omega t $, ed esse risultano \alert{in fase}.
\begin{figure}
\begin{tikzpicture}[xscale=.5,yscale=1]
\node [above,red] at (.5*pi,.2) {{\scriptsize $ i(t) $}};
\node [below,cyan] at (1.5*pi,-1.5) {{\scriptsize $ \fem(t) $}};
\node [below] at (2.5*pi,0) {{\scriptsize $ t $}};
\node [left] at (0,2) {{\scriptsize $ i(t)/\fem(t) $}};
\node [above] at (2*pi,0) {{\scriptsize $ T $}};
\draw [->] (-1,0) -- (2.5*pi,0);
\draw [->] (0,-2) -- (0,2);
\draw[smooth, red, thick, domain=0:2*pi] plot (\x, {sin(\x r)});
\draw[smooth, cyan, thick, domain=0:2*pi] plot (\x, {1.5*sin(\x r)});
\end{tikzpicture}
\end{figure}
\begin{center}
$ \fem(t) = \femm \cdot \sin (\omega t) $
~~~~~   $ i(t) = i_0 \cdot \sin (\omega t) $
\end{center}
\end{frame}





\begin{frame}
\frametitle{Esercizio}
\begin{exampleblock}{Circuito ohmico}
  \small{
    \begin{columns}
    \begin{column}{.4\textwidth}
      In un circuito puramente ohmico, un resistore di $ 100 \, \Omega $ è collegato ad un generatore che fornisce una $ \fem $ alternata di valore massimo $ 240 \, V $ e una frequenza di $ 50,0 \, Hz $.
    \end{column}
    \begin{column}{.4\textwidth}
      \begin{figure}
        \ctikzset{bipoles/length=.7cm}
        \begin{circuitikz}[scale=0.6]
        \draw (0,0) to[vco, a=\scriptsize $ \fem(t) $] (4,0);
        \draw (0,2) to[short] (0,0);
        \draw (0,2) to[R, l=\scriptsize $ R $] (4,2);
        \draw (4,2) to[short] (4,0);
        \end{circuitikz}
      \end{figure}
    \end{column}
    \end{columns}
    \begin{itemize}
      \item Calcola la pulsazione $ \omega $ della $ \fem $.\hspace*{\fill}[$ 314 \, rad/s $]
      \item Calcola il valore massimo della corrente che circola nel resistore.\hspace*{\fill}[$ 2,40 \, A $]
      \item Scrivi l'espressione della $ \fem $ in funzione del tempo.
    \end{itemize}}
\end{exampleblock}
\end{frame}






\section{Induttivo}

\begin{frame}
\frametitle{Circuito induttivo}
\begin{figure}
\ctikzset{bipoles/length=.7cm}
\begin{circuitikz}[scale=0.5]
\draw (0,0) to[vco, a=$ \fem(t) $] (6,0);
\draw (0,4) to[short] (0,0);
\draw (0,4) to[L, l=$ L $] (6,4);
\draw (6,4) to[short] (6,0);
\end{circuitikz}
\end{figure}
Utilizzando la legge delle maglie (e un po' di goniometria) otteniamo che:
\begin{center}
\colorbox{blue!30}{$ i(t) = \dfrac{\femm}{\omega L} \cdot \sin \left( \omega t - \dfrac{\pi}{2} \right) = i_0 \cdot \sin \left( \omega t - \dfrac{\pi}{2} \right)$} 
\end{center}
\end{frame}

\begin{frame}
\frametitle{Grafico delle funzioni}
La fase di entrambe le onde è diversa e la corrente oscilla in \alert{ritardo di un quarto di periodo}, a causa dei fenomeni di autoinduzione.
\begin{figure}
\begin{tikzpicture}[xscale=.5,yscale=1]
\node [above,red] at (1*pi,1) {{\scriptsize $ i(t) $}};
\node [below,cyan] at (1.5*pi,-1.5) {{\scriptsize $ \fem(t) $}};
\node [below] at (3*pi,0) {{\scriptsize $ t $}};
\node [left] at (0,2) {{\scriptsize $ i(t)/\fem(t) $}};
\node [above] at (2*pi,0) {{\scriptsize $ T $}};
\node [below] at (.5*pi,0) {{\scriptsize $ \dfrac{T}{4} $}};
\draw [->] (-1,0) -- (3*pi,0);
\draw [->] (0,-2) -- (0,2);
\draw[smooth, red, thick, domain=0.5*pi:2.5*pi] plot (\x, {sin((\x - .5*pi)  r )});
\draw[smooth, red, dashed, thick, domain=0:.5*pi] plot (\x, {sin((\x - .5*pi)  r )});
\draw[smooth, cyan, thick, domain=0:2*pi] plot (\x, {1.5*sin(\x r)});
\draw[smooth, cyan, dashed, thick, domain=2*pi:2.5*pi] plot (\x, {1.5*sin(\x r)});
\end{tikzpicture}
\end{figure}
\begin{center}
$ \fem(t) = \femm \cdot \sin (\omega t) $
~~~~~   $ i(t) = i_0 \cdot \sin \left( \omega t - \dfrac{\pi}{2} \right) $
\end{center}
\end{frame}



\begin{frame}
\frametitle{Esercizio}
\begin{exampleblock}{Circuito induttivo}
  \small{
    \begin{columns}
    \begin{column}{.4\textwidth}
      Un circuito puramente induttivo contiene una bobina con un'induttanza di $ 4,47 \, mH $ e un generatore che mantiene una $ \fem $ efficace di $ 16,0 \, V $ e una frequenza di $ 50,0 \, Hz $.
    \end{column}
    \begin{column}{.4\textwidth}
      \begin{figure}
        \ctikzset{bipoles/length=.7cm}
        \begin{circuitikz}[scale=0.6]
        \draw (0,0) to[vco, a=\scriptsize $ \fem(t) $] (4,0);
        \draw (0,2) to[short] (0,0);
        \draw (0,2) to[L, l=\scriptsize $ L $] (4,2);
        \draw (4,2) to[short] (4,0);
        \end{circuitikz}
      \end{figure}
    \end{column}
    \end{columns}
    \begin{itemize}
      \item Determina l'ampiezza della corrente oscillante presente nel circuito.\hspace*{\fill}[$ 16,1 \, A $]
      \item Calcola il valore efficace della corrente elettrica.\hspace*{\fill}[$ 11,4 \, A $]
    \end{itemize}}
\end{exampleblock}
\end{frame}



\section{Capacitivo}


\begin{frame}
\frametitle{Circuito capacitivo}
\begin{figure}
\ctikzset{bipoles/length=.7cm}
\begin{circuitikz}[scale=0.5]
\draw (0,0) to[vco, a=$ \fem(t) $] (6,0);
\draw (0,4) to[short] (0,0);
\draw (0,4) to[C, l=$ C $] (6,4);
\draw (6,4) to[short] (6,0);
\end{circuitikz}
\end{figure}
Con un procedimento analogo a quello per il circuito induttivo, otteniamo che:
\begin{center}
\colorbox{blue!30}{$ i(t) = \omega C \femm \cdot \sin \left( \omega t + \dfrac{\pi}{2} \right) = i_0 \cdot \sin \left( \omega t + \dfrac{\pi}{2} \right)$} 
\end{center}
\end{frame}

\begin{frame}
\frametitle{Grafico delle funzioni}
La fase di entrambe le onde è diversa e la corrente oscilla in \alert{anticipo di un quarto di periodo}.
\begin{figure}
\begin{tikzpicture}[xscale=.5,yscale=1]
\node [above,red] at (0*pi,1) {{\scriptsize $ i(t) $}};
\node [below,cyan] at (1.5*pi,-1.5) {{\scriptsize $ \fem(t) $}};
\node [below] at (2.5*pi,0) {{\scriptsize $ t $}};
\node [left] at (0,2) {{\scriptsize $ i(t)/\fem(t) $}};
\node [above] at (2*pi,0) {{\scriptsize $ T $}};
\node [below] at (-.5*pi,0) {{\scriptsize $ -\dfrac{T}{4} $}};
\draw [->] (-2,0) -- (2.5*pi,0);
\draw [->] (0,-2) -- (0,2);
\draw[smooth, red, thick, domain=-.5*pi:1.5*pi] plot (\x, {sin((\x + .5*pi)  r )});
\draw[smooth, red, dashed, thick, domain=1.5*pi:2*pi] plot (\x, {sin((\x + .5*pi)  r )});
\draw[smooth, cyan, thick, domain=0:2*pi] plot (\x, {1.5*sin(\x r)});
\draw[smooth, cyan, dashed, thick, domain=-.5*pi:0] plot (\x, {1.5*sin(\x r)});
\end{tikzpicture}
\end{figure}
\begin{center}
$ \fem(t) = \femm \cdot \sin (\omega t) $
~~~~~   $ i(t) = i_0 \cdot \sin \left( \omega t + \dfrac{\pi}{2} \right) $
\end{center}
\end{frame}


\begin{frame}
\frametitle{Esercizio}
\begin{exampleblock}{Circuito capacitivo}
  \small{
    Un condensatore e un avvolgimento fanno passare la stessa corrente massima quando sono alimentati dalla stessa differenza di potenziale alternata.
    \begin{itemize}
      \item Stabilisci la relazione che intercorre tra la capacità del condensatore e l'induttanza dell'avvolgimento.
    \end{itemize}}
\end{exampleblock}\pause

~

L'esercizio dice che $ i_{0 ~ ind} = i_{0 ~ cap} $, e quindi:
\begin{center}
  $ \dfrac{\femm}{\omega L} = \omega C \femm $
\end{center}\pause
Semplificando possiamo ottenere, ad esempio:
\begin{center}
  $ LC = \dfrac{1}{\omega^2} $
\end{center}
\end{frame}


\section{RLC}

\begin{frame}
\frametitle{Circuito RLC in serie}
\begin{figure}
\ctikzset{bipoles/length=.7cm}
\begin{circuitikz}[scale=0.5]
\draw (0,0) to[vco, a=$ \fem(t) $] (9,0);
\draw (0,3) to[short] (0,0);
\draw (0,3) to[R, l=$ R $] (3,3);
\draw (3,3) to[L, l=$ L $] (6,3);
\draw (6,3) to[C, l=$ C $] (9,3);
\draw (9,3) to[short] (9,0);
\end{circuitikz}
\end{figure}
Tra i valori efficaci della $ \fem $ e della corrente esiste una proporzionalità diretta data da:
\[ f_{em~eff} = Z \cdot i_{eff} \]
Notiamo come questa relazione sia sovrapponibile alla prima legge di Ohm, per cui $ \Delta V = R \cdot i $.
\end{frame}

\begin{frame}
\frametitle{Impedenza}
La costante di proporzionalità $ Z $ è detta \alert{impedenza}, si misura in $ \Omega $ e generalizza $ R $, prendendo i considerazione i fenomeni induttivi.

\begin{center}
\colorbox{blue!30}{$\displaystyle Z = \sqrt{R^2 - \left( \omega L - \frac{1}{\omega C} \right)^2} $} 
\end{center}

\begin{columns}
\begin{column}{0.45\textwidth}
\begin{center}
$ R \, [\Omega] $

~

resistenza del circuito
\end{center}
\end{column}
\begin{column}{0.45\textwidth}
\begin{center}
$ \displaystyle\omega L - \frac{1}{\omega C} \, [\Omega] $

~

reattanza del circuito
\end{center}
\end{column}
\end{columns}
\end{frame}





\begin{frame}
\frametitle{Esercizio}
\begin{exampleblock}{Circuito RLC}
  \small{
    Un circuito RLC in serie è composto da una resistenza da $ 250 \, \Omega $, una bobina da $ 3,40 \, mH $ e un condensatore da $ 98,0 \, nF $. Il circuito è collegato ad una alimentazione alternata con un valore efficace di $ 14,0 \, V $ e una frequenza di $ 5,00 \, kHz $.
    \begin{figure}
      \ctikzset{bipoles/length=.7cm}
      \begin{circuitikz}[scale=0.4]
      \draw (0,0) to[vco, a=\scriptsize $ \fem(t) $] (9,0);
      \draw (0,3) to[short] (0,0);
      \draw (0,3) to[R, l=\scriptsize $ R $] (3,3);
      \draw (3,3) to[L, l=\scriptsize $ L $] (6,3);
      \draw (6,3) to[C, l=\scriptsize $ C $] (9,3);
      \draw (9,3) to[short] (9,0);
      \end{circuitikz}
    \end{figure}
    Determina l'intensità efficace della corrente elettrica che fluisce nel circuito.\hspace*{\fill}[$ 4,22 \times 10^{-2} \, A $]}
\end{exampleblock}
\end{frame}




\section{LC}


\begin{frame}
\frametitle{Circuito LC}
Un circuito LC è un circuito senza generatore, in cui sono collegati in serie un condensatore e un induttore.
\begin{figure}
  \ctikzset{bipoles/length=.7cm}
  \begin{circuitikz}[scale=0.6]
  \draw (0,0) to[L, a=$ L $] (6,0);
  \draw (0,3) to[short] (0,0);
  \draw (6,3) to[C, a=$ C $] (0,3);
  \draw (6,3) to[short] (6,0);
  \end{circuitikz}
\end{figure}\pause
Per questo circuito è possibile dimostrare con le equazioni differenziali che \alert<2>{la corrente oscilla} secondo la funzione:
\begin{center}
  \colorbox{blue!30}{$ i(t) = i_0 \cos (\omega t) $} ~~~ con $ i_0 = \dfrac{1}{\sqrt{LC}} $ 
\end{center}
\end{frame}

\end{document}
