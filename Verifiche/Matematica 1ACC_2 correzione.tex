\documentclass[12pt]{exam}
\usepackage[T1]{fontenc}
\usepackage[utf8]{inputenc}
\usepackage{lmodern}
\usepackage[italian]{babel}

\usepackage[margin=1in]{geometry}
\usepackage{amsmath,amssymb}
\usepackage{multicol}

\usepackage{tikz}
\usetikzlibrary{calc}
\def\angolo[#1](#2)(#3:#4:#5)% Syntax: [draw options] (center) (initial angle:final angle:radius)
    { \draw[#1] ($(#2)+({#5*cos(#3)},{#5*sin(#3)})$) arc (#3:#4:#5); }

\newcommand{\class}{Matematica 1 ACC}     %Fisica 5 LS
\newcommand{\examnum}{Verifica 2}               %Verifica 2
\newcommand{\examdate}{4 novembre 2024}                      %29/10/2018
\newcommand{\timelimit}{1 ora}                   %2 ore
\newcommand{\topic}{Unità di misura ed equivalenze}                    %Campo elettrico e campo magnetico

\pagestyle{head}
\firstpageheader{\class}{\examnum\ - Pagina \thepage\ di \numpages}{\examdate}
\runningheader{\class}{\examnum\ - Pagina \thepage\ di \numpages}{\examdate}
\runningheadrule


\begin{document}

\noindent
\begin{tabular*}{\textwidth}{l @{\extracolsep{\fill}} r @{\extracolsep{6pt}} r}
 \textbf{\topic} && \textbf{Durata: \timelimit}\\
 &&\\
 \textbf{Nome:} \makebox[2.9in]{\hrulefill} && \\
\end{tabular*}
\rule[2ex]{\textwidth}{2pt}


% Punteggio totale: \numpoints.
\begin{center}
\addpoints
% \gradetable[v][questions]
\multirowgradetable{1}[questions]
\end{center}

\noindent
\rule[2ex]{\textwidth}{2pt}

\begin{questions}





%DOMANDA SECCA
\question[35] Rispondi alle seguenti domande di teoria.
\renewcommand{\labelenumi}{(\Alph{enumi})}
\begin{enumerate}
    \item Elenca almeno tre strumenti di misurazione della lunghezza.
    \item Che cosa è la sensibilità di uno strumento di misura?
    \item Elenca almeno tre unità di misura derivate da quelle fondamentali.
    \item Che cosa sono i multipli di una unità di misura?
    \item Quali sono i sottomultipli del metro?
    \item Che cosa accade al valore numerico di una misura se passo ad una unità di misura più piccola?
\end{enumerate}
\addpoints



\question[20] La figura in basso rappresenta le misure di una stanza, misurate però con unità di misura diverse. Disegna la piantina della stanza, convertendo tutte le misure in una unità di misura a tua scelta. Calcola il valore della misura ignota.
\begin{figure}[!ht]\centering
\begin{tikzpicture}[scale=1.3]
\node [below] at (5,0) {$ 10 \, m $};
\node [below] at (6,3) {$ 2 \, m $};
\node [right] at (10,2.5) {$ 5 \, m $};
\node [right] at (7,4) {$ 2 \, m $};
\node [left] at (0,2.5) {$ 5 \, m $};
\node [left] at (5,4) {$ 2 \, m $};
\node [above] at (2.5,5) {$ 5 \, m $};
\node [above] at (8.5,5) {$ 3 \, m $};
\draw [thick] (0,0) -- (10,0) -- (10,5) -- (7,5) -- (7,3) -- (5,3) -- (5,5) -- (0,5) -- (0,0);
\end{tikzpicture}
\end{figure}
\addpoints


\newpage
%DOMANDA DIVISA IN PIÙ RICHIESTE INDICATE CON (a), (b)...
\question[15] Converti le seguenti misure in metri:
\noaddpoints % to omit double points count
\begin{multicols}{2}
\begin{parts}
    \part $ 65 \, dm  = 6,5 \, m $

    ~
    \part $ 0,987 \, mm  = 0,000987 \, m $

    ~
    \part $ 320,6 \, cm = 3,26 \, m $

    ~

    ~
\end{parts}
\end{multicols}
\addpoints


\question[15] Converti le seguenti misure in L:
\noaddpoints % to omit double points count
\begin{multicols}{2}
\begin{parts}
    \part $ 1323 \, mL = 1,323 \, L $

    ~
    \part $ 98 \, dL = 9,8 \, L $

    ~
    \part $ 0,036 \, daL = 0,36 \, L $

    ~

    ~
\end{parts}
\end{multicols}
\addpoints

\question[15] Converti le seguenti misure nell'unità di misura indicata:
\noaddpoints % to omit double points count
\begin{multicols}{2}
\begin{parts}
    \part $ 1,7 \, kW = 1700 \, W $

    ~
    \part $ 320 \, mA = 0,320 \, A $

    ~
    \part $ 360 \, kBit = 360000 \, Bit $

    ~

    ~
\end{parts}
\end{multicols}
\addpoints





\end{questions}

\end{document}
