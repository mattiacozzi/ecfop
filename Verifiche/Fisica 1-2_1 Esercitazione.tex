\documentclass[12pt]{exam}
\usepackage[T1]{fontenc}
\usepackage[utf8]{inputenc}
\usepackage{lmodern}
\usepackage[italian]{babel}

\usepackage[margin=1in]{geometry}
\usepackage{amsmath,amssymb}
\usepackage{multicol}
\usepackage{mathrsfs}
\usepackage{tikz}
\usetikzlibrary{calc}

\def\centerarc[#1](#2)(#3:#4:#5)% Syntax: [draw options] (center) (initial angle:final angle:radius)
    { \draw[#1] ($(#2)+({#5*cos(#3)},{#5*sin(#3)})$) arc (#3:#4:#5); }

\newcommand{\class}{Fisica 1-2}
\newcommand{\examnum}{Verifica 1 - Esercitazione}
\newcommand{\examdate}{26 ottobre 2023}
\newcommand{\timelimit}{1 ora}
\newcommand{\topic}{Calcolo scientifico}

\pagestyle{head}
\firstpageheader{\class}{\examnum\ - Pagina \thepage\ di \numpages}{\examdate}
\runningheader{\class}{\examnum\ - Pagina \thepage\ di \numpages}{\examdate}
\runningheadrule


\begin{document}

\noindent
\begin{tabular*}{\textwidth}{l @{\extracolsep{\fill}} r @{\extracolsep{6pt}} r}
 \textbf{\topic} && \textbf{Durata: \timelimit}\\
 &&\\
 \textbf{Nome:} \makebox[2.9in]{\hrulefill} && \\
\end{tabular*}
\rule[2ex]{\textwidth}{2pt}

% Punteggio totale: \numpoints.
% \begin{center}
% \addpoints
% % \gradetable[v][questions]
% \multirowgradetable{1}[questions]
% \end{center}

% \noindent
% \rule[2ex]{\textwidth}{2pt}

\begin{questions}




\question Converti i seguenti numeri in notazione scientifica.
\addpoints
\begin{multicols}{4}
\begin{parts}
\part $ 81969 $
\part $ 894 \times 10^{-6} $
\part $ 22,44 \times 10^{8} $
\part $ 23,00 $
\end{parts}
\end{multicols}

\question Converti le seguenti misure in metri, usando la notazione scientifica.
\noaddpoints % to omit double points count
\begin{multicols}{4}
\begin{parts}
\part $ 349,00 \, mm $
\part $ 1500 \, m $
\part $ 456 \times 10^{4} \, nm $
\part $ 11 \times 10^{-8} \, mm $
\end{parts}
\end{multicols}
\addpoints



\question Calcola il valore di $ P $:
\[ P = \frac{F \cdot s}{\Delta t} \]
\begin{multicols}{2}
\begin{itemize}
    \item $ F = 5000 \, N $;
    \item $ s = 1,3 \, m $;
    \item $ \Delta t = 4,10 \times 10^{2} \, s $.
\end{itemize}
\end{multicols}
\addpoints





\question Calcola il valore di $ \alpha $:
\[ \lambda = \frac{\Delta \ell}{\ell_0 \cdot \Delta T} \]
\begin{multicols}{2}
\begin{itemize}
    \item $ \Delta V = 3,5 \times 10^{-6} \, m $;
    \item $ V_0 = 7,0 \times 10^{-4} \, m $;
    \item $ \Delta T = 100 \, K $.
\end{itemize}
\end{multicols}
\addpoints




\question Calcola il valore di $ F $:
\[ F = k_0 \cdot \frac{q_1 \cdot q_2}{r^2} \]
\begin{multicols}{2}
\begin{itemize}
    \item $ k_0 = 8,99 \times 10^{9} \, \frac{Nm^2}{C^2} $;
    \item $ q_1 = 2,00 \, \mu C $;
    \item $ q_2 = 8,00 \, \mu C $;
    \item $ r = 4,00 \, mm $.
\end{itemize}
\end{multicols}
\addpoints



\end{questions}

\end{document}
