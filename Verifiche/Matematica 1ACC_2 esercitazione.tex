\documentclass[12pt]{exam}
\usepackage[T1]{fontenc}
\usepackage[utf8]{inputenc}
\usepackage{lmodern}
\usepackage[italian]{babel}

\usepackage[margin=1in]{geometry}
\usepackage{amsmath,amssymb}
\usepackage{multicol}

\usepackage{tikz}
\usetikzlibrary{calc}
\def\angolo[#1](#2)(#3:#4:#5)% Syntax: [draw options] (center) (initial angle:final angle:radius)
    { \draw[#1] ($(#2)+({#5*cos(#3)},{#5*sin(#3)})$) arc (#3:#4:#5); }

\newcommand{\class}{Matematica 1 ACC}     %Fisica 5 LS
\newcommand{\examnum}{Esercitazione 1}               %Verifica 2
\newcommand{\examdate}{28 ottobre 2024}                      %29/10/2018
\newcommand{\timelimit}{1 ora}                   %2 ore
\newcommand{\topic}{Unità di misura ed equivalenze}                    %Campo elettrico e campo magnetico

\pagestyle{head}
\firstpageheader{\class}{\examnum\ - Pagina \thepage\ di \numpages}{\examdate}
\runningheader{\class}{\examnum\ - Pagina \thepage\ di \numpages}{\examdate}
\runningheadrule


\begin{document}

\noindent
\begin{tabular*}{\textwidth}{l @{\extracolsep{\fill}} r @{\extracolsep{6pt}} r}
 \textbf{\topic} && \textbf{Durata: \timelimit}\\
 &&\\
 \textbf{Nome:} \makebox[2.9in]{\hrulefill} && \\
\end{tabular*}
\rule[2ex]{\textwidth}{2pt}



\begin{questions}





%DOMANDA SECCA
\question Rispondi alle seguenti domande di teoria.
\renewcommand{\labelenumi}{(\Alph{enumi})}
\begin{enumerate}
    \item Elenca almeno tre strumenti di misurazione del tempo.
    \item Elenca almeno due strumenti di misurazione del volume.
    \item Elenca almeno tre strumenti di misurazione della lunghezza.
    \item Elenca almeno due strumenti di misurazione della massa.
    \item Che cosa è la sensibilità di uno strumento di misura?
    \item Che cosa è la portata di uno strumento di misura?
    \item Elenca 15 cose che si possono misurare e scrivi l'unità di misura necessaria.
    \item Che cosa è il Sistema Internazionale?
    \item Elenca almeno tre unità di misura derivate da quelle fondamentali.
    \item Che cosa sono i multipli di una unità di misura?
    \item Che cosa sono i sottomultipli di una unità di misura?
    \item Quali sono i sottomultipli del metro?
    \item Quali sono i multipli del grammo?
    \item Quali sono i sottomultipli del litro?
    \item Che cosa è un'equivalenza tra unità di misura?
    \item Che cosa accade al valore numerico di una misura se passo ad una unità di misura più piccola?
    \item Che cosa accade al valore numerico di una misura se passo ad una unità di misura più grande?
    \item Possiamo convertire da grammi a metri? Perché?
\end{enumerate}
\addpoints

\newpage

\question Completa la figura con i dati mancanti:
\begin{figure}[!ht]\centering
\begin{tikzpicture}[scale=.7]
    \draw [thick,fill=gray!30] (0,0) -- (4,0) -- (3.3,2) -- (0.7,2) -- (0,0);
    \node [right] at (4,1) {$=  5 \, kg $};
\begin{scope}[shift={(12,0)}]  
    \draw [thick,fill=gray!30] (0,0) -- (4,0) -- (3.3,2) -- (0.7,2) -- (0,0);
    \node [right] at (4,1) {$=  \qquad\qquad \, g $};
\end{scope}
\begin{scope}[shift={(0,-5)}]  
    \draw [thick,fill=gray!30] (0,0) -- (4,0) -- (3.3,2) -- (0.7,2) -- (0,0);
    \node [right] at (4,1) {$=  \qquad\qquad \, hg $};
\end{scope}
\begin{scope}[shift={(12,-5)}]  
    \draw [thick,fill=gray!30] (0,0) -- (4,0) -- (3.3,2) -- (0.7,2) -- (0,0);
    \draw [thick,fill=gray!30] (0,2) -- (4,2) -- (3.3,4) -- (0.7,4) -- (0,2);
    \node [right] at (4,1) {$=  \qquad\qquad \, g $};
\end{scope}
\begin{scope}[shift={(0,-10)}]
    \draw [thick,fill=gray!30] (0,0) -- (2,0) -- (2,2) -- (0.7,2) -- (0,0);
    \node [right] at (4,1) {$=  \qquad\qquad \, cg $};
\end{scope}
\begin{scope}[shift={(12,-10)}]  
    \draw [thick,fill=gray!30] (0,0) -- (4,0) -- (3.3,2) -- (0.7,2) -- (0,0);
    \draw [thick,fill=gray!30] (0,2) -- (2,2) -- (2,4) -- (0.7,4) -- (0,2);
    \node [right] at (4,1) {$=  \qquad\qquad \, mg $};
\end{scope}
\end{tikzpicture}
\end{figure}
\addpoints




\question La figura in basso rappresenta le misure di una stanza, misurate però con unità di misura diverse. Disegna la piantina della stanza, convertendo tutte le misure in una unità di misura a tua scelta. Calcola il valore della misura ignota.
\begin{figure}[!ht]\centering
\begin{tikzpicture}[scale=1.3]
\node [below] at (5,0) {$ 10 \, m $};
\node [below] at (6,3) {$ 200 \, cm $};
\node [right] at (10,2.5) {$ 5000 \, mm $};
\node [right] at (7,4) {$ 0,20 \, dam $};
\node [left] at (0,2.5) {$ 0,0050 \, km $};
\node [left] at (5,4) {$ 20 \, dm $};
\node [above] at (2.5,5) {$ 0,050 \, hm $};
\node [above] at (8.5,5) {?};
\draw [thick] (0,0) -- (10,0) -- (10,5) -- (7,5) -- (7,3) -- (5,3) -- (5,5) -- (0,5) -- (0,0);
\end{tikzpicture}
\end{figure}
\addpoints



\newpage


%DOMANDA DIVISA IN PIÙ RICHIESTE INDICATE CON (a), (b)...
\question Converti le seguenti misure in metri:
\noaddpoints % to omit double points count
\begin{multicols}{2}
\begin{parts}
    \part $ 1348 \, mm $

    ~
    \part $ 0,034 \, cm $

    ~
    \part $ 150 \, km $

    ~
    \part $ 87 \, dm $

    ~
    \part $ 0,234 \, mm $

    ~
    \part $ 0,045 \, dam $

    ~
    \part $ 200,5 \, cm $

    ~
    \part $ 780 \, mm $

    ~
    \part $ 0,0054 \, km $

    ~
    \part $ 197 \, cm $

    ~
\end{parts}
\end{multicols}
\addpoints


\question Converti le seguenti misure in cm:
\noaddpoints % to omit double points count
\begin{multicols}{2}
\begin{parts}
    \part $ 1348 \, mm $

    ~
    \part $ 87 \, dm $

    ~
    \part $ 0,234 \, mm $

    ~
    \part $ 0,045 \, dam $

    ~
    \part $ 780 \, mm $

    ~
    \part $ 0,0054 \, km $

    ~
\end{parts}
\end{multicols}
\addpoints

\question Converti le seguenti misure nell'unità di misura indicata:
\noaddpoints % to omit double points count
\begin{multicols}{2}
\begin{parts}
    \part $ 1,5 \, kW = \qquad\qquad \, W $

    ~
    \part $ 200 \, mA = \qquad\qquad \, A $

    ~
    \part $ 180 \, kcal = \qquad\qquad \, cal $

    ~
    \part $ 290 \, kBit = \qquad\qquad \, Bit $

    ~
    \part $ 12 \, V = \qquad\qquad \, mV $

    ~
    \part $ 500  \, mT = \qquad\qquad \, T $

    ~
\end{parts}
\end{multicols}
\addpoints





\end{questions}

\end{document}
